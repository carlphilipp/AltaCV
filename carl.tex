%%%%%%%%%%%%%%%%%
% This is an sample CV template created using altacv.cls
% (v1.7.4, 30 Jul 2025) written by LianTze Lim (liantze@gmail.com), based on the
% CV created by BusinessInsider at http://www.businessinsider.my/a-sample-resume-for-marissa-mayer-2016-7/?r=US&IR=T
%
%% It may be distributed and/or modified under the
%% conditions of the LaTeX Project Public License, either version 1.3
%% of this license or (at your option) any later version.
%% The latest version of this license is in
%%    http://www.latex-project.org/lppl.txt
%% and version 1.3 or later is part of all distributions of LaTeX
%% version 2003/12/01 or later.
%%%%%%%%%%%%%%%%

%% Use the "normalphoto" option if you want a normal photo instead of cropped to a circle
% \documentclass[10pt,a4paper,withhypeper,normalphoto]{altacv}

\documentclass[10pt,a4paper,withhyper]{altacv}
%% AltaCV uses the fontawesome5 and simpleicons packages.
%% See http://texdoc.net/pkg/fontawesome5 and http://texdoc.net/pkg/simpleicons for full list of symbols.

% Change the page layout if you need to
\geometry{left=1.25cm,right=1.25cm,top=1.5cm,bottom=1.5cm,columnsep=1.2cm}

% The paracol package lets you typeset columns of text in parallel
\usepackage{paracol}


% Change the font if you want to, depending on whether
% you're using pdflatex or xelatex/lualatex
% WHEN COMPILING WITH XELATEX PLEASE USE
% xelatex -shell-escape -output-driver="xdvipdfmx -z 0" mmayer.tex
\iftutex
  % If using xelatex or lualatex:
  \setmainfont{Lato}
\else
  % If using pdflatex:
  \usepackage[default]{lato}
\fi

% Change the colours if you want to
\definecolor{VividPurple}{HTML}{3E0097}
\definecolor{SlateGrey}{HTML}{2E2E2E}
\definecolor{LightGrey}{HTML}{666666}
% \colorlet{name}{black}
% \colorlet{tagline}{PastelRed}
\colorlet{heading}{VividPurple}
\colorlet{headingrule}{VividPurple}
% \colorlet{subheading}{PastelRed}
\colorlet{accent}{VividPurple}
\colorlet{emphasis}{SlateGrey}
\colorlet{body}{LightGrey}

% Change some fonts, if necessary
% \renewcommand{\namefont}{\Huge\rmfamily\bfseries}
% \renewcommand{\personalinfofont}{\footnotesize}
% \renewcommand{\cvsectionfont}{\LARGE\rmfamily\bfseries}
% \renewcommand{\cvsubsectionfont}{\large\bfseries}

% Change the bullets for itemize and rating marker
% for \cvskill if you want to
\renewcommand{\cvItemMarker}{{\small\textbullet}}
\renewcommand{\cvRatingMarker}{\faCircle}
% ...and the markers for the date/location for \cvevent
% \renewcommand{\cvDateMarker}{\faCalendar*[regular]}
% \renewcommand{\cvLocationMarker}{\faMapMarker*}


% If your CV/résumé is in a language other than English,
% then you probably want to change these so that when you
% copy-paste from the PDF or run pdftotext, the location
% and date marker icons for \cvevent will paste as correct
% translations. For example Spanish:
% \renewcommand{\locationname}{Ubicación}
% \renewcommand{\datename}{Fecha}


%% Use (and optionally edit if necessary) this .tex if you
%% want to use an author-year reference style like APA(6)
%% for your publication list
% \input{pubs-authoryear.cfg}

%% Use (and optionally edit if necessary) this .tex if you
%% want an originally numerical reference style like IEEE
%% for your publication list
\input{pubs-num.cfg}

%% sample.bib contains your publications
\addbibresource{sample.bib}

\begin{document}
\name{Carl-Philipp Harmant}
\tagline{Lead Software Engineer}
% Cropped to square from https://en.wikipedia.org/wiki/Marissa_Mayer#/media/File:Marissa_Mayer_May_2014_(cropped).jpg, CC-BY 2.0
%% You can add multiple photos on the left or right
%\photoR{2.5cm}{mmayer-wikipedia-cc-by-2_0}
% \photoL{2cm}{Yacht_High,Suitcase_High}
%\personalinfo{%
  % Not all of these are required!
  %\email{mmayer@yahoo-inc.com}
  %\phone{000-00-0000}
%  \mailaddress{Address, Street, 00000 County}
%  \location{Sunnyvale, CA}
%  \linkedin{marissamayer}
  % \twitter{@marissamayer}
%  \xtwitter{@marissamayer}
%  \homepage{marissamayr.tumblr.com}
%   \github{github.com/mmayer} % I'm just making this up though.
%   \orcid{0000-0000-0000-0000} % Obviously making this up too.
  %% You can add your own arbitrary detail with
  %% \printinfo{symbol}{detail}[optional hyperlink prefix]
  %% \printinfo{\faPaw}{Hey ho!}
  %% Or you can declare your own field with
  %% \NewInfoFiled{fieldname}{symbol}[optional hyperlink prefix] and use it:
  % \NewInfoField{gitlab}{\faGitlab}[https://gitlab.com/]
  % \gitlab{your_id}
	%%
  %% For services and platforms like Mastodon where there isn't a
  %% straightforward relation between the user ID/nickname and the hyperlink,
  %% you can use \printinfo directly e.g.
  % \printinfo{\faMastodon}{@username@instace}[https://instance.url/@username]
  %% But if you absolutely want to create new dedicated info fields for
  %% such platforms, then use \NewInfoField* with a star:
  % \NewInfoField*{mastodon}{\faMastodon}
  %% then you can use \mastodon, with TWO arguments where the 2nd argument is
  %% the full hyperlink.
  % \mastodon{@username@instance}{https://instance.url/@username}
%}

\makecvheadercusto

%% Depending on your tastes, you may want to make fonts of itemize environments slightly smaller
\AtBeginEnvironment{itemize}{\small}

%% Set the left/right column width ratio to 6:4.
\columnratio{0.72}

% Start a 2-column paracol. Both the left and right columns will automatically
% break across pages if things get too long.
\begin{paracol}{2}

\cvsection{synopsis}

{\small 

Enthusiastic and curious Software Engineer dedicated to building thoughtful, high-quality software. My main toolkit includes Java, Kotlin, and modern architectural patterns that enable clean, maintainable codebases. I’m driven by problem-solving, learning, and contributing to the developer community — whether through my open-source projects on GitHub or sharing knowledge via technical writing, including my article on Clean Architecture. I love exploring new technologies and continuously improving both my craft and the systems I help build.

}

\cvsection{Experience}

\cveventnew{Senior Software Engineer}{The Walt Disney Company}{June 2022 -- Present}{Chicago, IL (Remote)}
\begin{itemize}
\item Led design and development of scalable, microservices-based backend systems and APIs (REST, GraphQL, and gRPC) powering Disney+, Hulu, and ESPN streaming platforms.
\item Delivered end-to-end solutions from architecture and design through deployment and production, ensuring reliability under high-concurrency traffic (8K+ RPS).
\item Drove system performance tuning, observability improvements, and CI/CD pipeline optimizations, cutting build and deployment times by 50\%.
\item Partnered with architects, DevOps, and security teams to strengthen cloud infrastructure, automation, and compliance best practices.
\item Provided technical leadership and mentorship, influencing architecture, code quality, and cross-team engineering practices.
\end{itemize}

\divider

\cveventnew{Senior Software Engineer}{Slalom}{July 2017 -- June 2022}{Chicago, IL}
\begin{itemize}
\item Designed and developed scalable backend systems and APIs for enterprise clients using Java and Spring Boot.
\item Collaborated with architects, solution owners, and cross-functional teams to deliver maintainable, production-grade solutions.
\item Championed code quality and testing practices (unit, integration, and end-to-end), improving reliability and maintainability.
\item Contributed to cloud-based architectures and CI/CD pipelines, supporting continuous delivery and high availability.
\item Mentored junior engineers and promoted engineering best practices across agile teams.
\end{itemize}

\divider

\cveventmultiple{Peapod}{Chicago, IL}
				{Senior Software Engineer}{Jan 2017 --  July 2017}
				{Software Engineer}{Sept 2015 --  Jan 2017}
\begin{itemize}		
\item Designed and built secure, RESTful integration services using Spring MVC, Apache Camel, Karaf, and ActiveMQ, enabling seamless communication between loyalty, coupon, and user account systems.
\item Championed TDD and automated testing (Serenity BDD) to improve reliability, maintainability, and long-term architecture quality across the platform.
\end{itemize}

\divider

\cveventnew{}{Previous roles}{}{}
\begin{itemize}
%\item Peapod
\item Pros: Implementation Consultant. (Feb 2012 -- Sept 2015 - Chicago, IL) Integrated and customized PROS CPQ solutions within enterprise systems using Java EE, web services, and custom plugins, contributing to full-stack delivery from UI design (HTML, CSS, JavaScript) to cloud deployment.
\item Fujitsu Technology Solution: Java Developer. (Apr. 2011 - Aug. 2011, Luxembourg) Engineered a custom Maven plugin to automate Arquillian test generation and dependency resolution via Java Reflection, significantly improving the company’s integration testing process and developer efficiency.
\end{itemize}


% \divider

% \cvevent{Product Engineer}{Google}{23 June 1999 -- 2001}{Palo Alto, CA}

% \begin{itemize}
% \item Joined the company as employe \#20 and female employee \#1
% \item Developed targeted advertisement in order to use user's search queries and show them related ads
% \end{itemize}

%\cvsection{A Day of My Life}

% Adapted from @Jake's answer from http://tex.stackexchange.com/a/82729/226
% \wheelchart{outer radius}{inner radius}{
% comma-separated list of value/text width/color/detail}
% Some ad-hoc tweaking to adjust the labels so that they don't overlap
%\hspace*{-1em}  %% quick hack to move the wheelchart a bit left
%\wheelchart{1.5cm}{0.5cm}{%
%  10/13em/accent!30/Sleeping \& dreaming about work,
%  25/9em/accent!60/Public resolving issues with Yahoo!\ investors,
%  5/11em/accent!10/\footnotesize\\[1ex]New York \& San Francisco Ballet Jawbone board member,
%  20/11em/accent!40/Spending time with family,
%  5/8em/accent!20/\footnotesize Business development for Yahoo!\ after the Verizon acquisition,
%  30/9em/accent/Showing Yahoo!\ \mbox{employees} that their work has meaning,
%  5/8em/accent!20/Baking cupcakes
%}

% use ONLY \newpage if you want to force a page break for
% ONLY the currentc column
%\newpage

%\cvsection{Publications}

%% Specify your last name(s) and first name(s) as given in the .bib to automatically bold your own name in the publications list.
%% One caveat: You need to write \bibnamedelima where there's a space in your name for this to work properly; or write \bibnamedelimi if you use initials in the .bib
%% You can specify multiple names, especially if you have changed your name or if you need to highlight multiple authors.
%\mynames{Lim/Lian\bibnamedelima Tze,
%  Wong/Lian\bibnamedelima Tze,
%  Lim/Tracy,
%  Lim/L.\bibnamedelimi T.}
%% MAKE SURE THERE IS NO SPACE AFTER THE FINAL NAME IN YOUR \mynames LIST

%\nocite{*}

%\printbibliography[heading=pubtype,title={\printinfo{\faBook}{Books}},type=book]

%\divider

%\printbibliography[heading=pubtype,title={\printinfo{\faFile*[regular]}{Journal Articles}}, type=article]

%\divider

%\printbibliography[heading=pubtype,title={\printinfo{\faUsers}{Conference Proceedings}},type=inproceedings]

%% Switch to the right column. This will now automatically move to the second
%% page if the content is too long.
\switchcolumn

%\cvsection{Life Philosophy}
%\begin{quote}
%``If you don't have any shadows, you're not standing in the light.''
%\end{quote}

\cvsection{Contact}

	\email{cp.harmant@gmail.com}
	
	\divider


	\phone{312-834-1637}
	
	\divider


	\location{Chicago, Illinois}
	
	\divider

	\linkedin{carlphilipp}
	
	\divider
	
	\github{carlphilipp}

%\cvsection{Most Proud of}

%\cvachievement{\faTrophy}{Courage I had}{to take a sinking ship and try to make it float}

%\divider

%\cvachievement{\faHeartbeat}{Persistence \& Loyalty}{I showed despite the hard moments and my willingness to stay with Yahoo after the acquisition}

%\divider

%\cvachievement{\faChartLine}{Google's Growth}{from a hundred thousand searches per day to over a billion}

%\divider

%\cvachievement{\faFemale}{Inspiring women in tech}{Youngest CEO on Fortune's list of 50 most powerful women}

\cvsection{Skills}

% Don't overuse these \cvtag boxes — they're just eye-candies and not essential. If something doesn't fit on a single line, it probably works better as part of an itemized list (probably inlined itemized list), or just as a comma-separated list of strengths.

\cvtag{Backend}
\cvtag{Architecture}
\cvtag{Distributed Systems}
\cvtag{Cloud}
\cvtag{Microservices}
\cvtag{CI/CD}
\cvtag{Agile}
\cvtag{Observability}


\divider\smallskip

\cvtag{Java}
\cvtag{Spring Boot}
\cvtag{GraphQL}
\cvtag{gRPC}
\cvtag{REST}
\cvtag{Datadog}

\cvtag{Docker}
\cvtag{K8s}
\cvtag{Jenkins}
\cvtag{Spinnaker}
\cvtag{AWS}



%\cvsection{Languages}

%\cvskill{English}{5}
%\divider

%\cvskill{Spanish}{4}
%\divider

%\cvskill{German}{3.5} %% supports X.5 values.


\cvsection{Education}

\cvevent{M.S.\ in Computer Science}{Exia.Cesi, France}{2013}{}

\divider

\cvevent{B.S.\ in Computer Science}{Exia.Cesi, France}{2011}{}


%\newpage

%\cvsection{Referees}

% \cvref{name}{email}{mailing address}
%\cvref{Prof.\ Alpha Beta}{Institute}{a.beta@university.edu}
%{Address Line 1\\Address Line 2}

%\divider

%\cvref{Prof.\ Gamma Delta}{Institute}{g.delta@university.edu}
%{Address Line 1\\Address Line 2}

\end{paracol}

\end{document}
